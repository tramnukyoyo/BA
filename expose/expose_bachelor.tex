\documentclass[12pt, a4paper, oneside]{article}
\usepackage[utf8]{inputenc}
\usepackage[ngerman]{babel}
\usepackage{csquotes}
\usepackage[hidelinks]{hyperref}
\usepackage{graphicx}
\usepackage[backend=biber,style=authoryear,sorting=nyt]{biblatex}
\usepackage{geometry}
\usepackage{setspace}
\usepackage{ragged2e}
\usepackage{fancyhdr}

% Margins
\geometry{
  top=2.5cm,
  bottom=2.5cm,
  left=2.5cm,
  right=2.5cm
}

% Bibliography
\addbibresource{references_h2.bib}

% Line spacing
\setstretch{1.5}

% Header/Footer
\pagestyle{fancy}
\fancyhead{}
\fancyfoot{}
\fancyfoot[C]{\thepage}
\renewcommand{\headrulewidth}{0pt}

% Title page setup
\title{
  \textbf{Wasserstoffinfrastruktur in globalen Lieferketten}\\
  \vspace{0.5cm}
  \normalsize Implementierungsstrategien und Herausforderungen\\
  \vspace{0.3cm}
  \normalsize Exposé für eine Bachelorarbeit
}
\author{Denny}
\date{\today}

\begin{document}

\maketitle

\thispagestyle{empty}

\newpage

\tableofcontents
\thispagestyle{empty}

\newpage
\setcounter{page}{1}

% ============================================================
% AUSGANGSLAGE
% ============================================================

\section*{Ausgangslage}
\addcontentsline{toc}{section}{Ausgangslage}

Die globale Energiewende erfordert die schnelle Dekarbonisierung von schwer zu elektrifizierenden Sektoren wie Stahlproduktion, Chemie und Langstreckentransport. Wasserstoff (H\textsubscript{2}) gilt dabei als Schlüsseltechnologie: Er kann als Energieträger für diese Sektoren dienen und in erneuerbaren Energiesystemen überschüssige Elektrizität speichern \cite{EUCommission2020}.

\subsection*{Die Implementierungslücke (2025)}

Die globale H\textsubscript{2}-Infrastruktur weist eine erhebliche Lücke zwischen Ankündigungen und Realisierung auf. Aktuelle Forschung zeigt, dass von den weltweit angekündigten H\textsubscript{2}-Kapazitäten im Jahr 2022 nur 2\% pünktlich realisiert wurden; 71\% verzeichneten Verzögerungen und 25\% wurden ganz aufgegeben \cite{Odenweller2025}. Dies unterstreicht das systematische Dilemma zwischen Infrastrukturinvestitionen und Nachfrageunsicherheit.

Konkrete Beispiele regionaler Initiativen verdeutlichen unterschiedliche Implementierungsansätze:

\begin{itemize}
  \item \textbf{Europäische Union:} Die 2020 verabschiedete Wasserstoffstrategie \cite{EUCommission2020} plant den Aufbau des ``Hydrogen Backbone'' mit bis zu 39.700 km Wasserstoff-Pipelines bis 2040. Investitionsbedarf: €24-42 Mrd. für Elektrolyseure und €220-340 Mrd. für erneuerbare Energiekapazitäten bis 2030.

  \item \textbf{Japan:} Die nationale H\textsubscript{2}-Strategie (2023) \cite{METI2023} setzt auf Importe von grünem Wasserstoff und Wasserstoffträgern (Ammoniak, LOHC) aus strategischen Partnern wie Australien und dem Nahen Osten.

  \item \textbf{Australien:} Mit reichlich vorhandener erneuerbarer Energie positioniert sich das Land als Exporteur von grünem Wasserstoff und Wasserstoffderivaten zu asiatischen Märkten.
\end{itemize}

\subsection*{Kostenstruktur und Wirtschaftlichkeit}

Die technische und wirtschaftliche Machbarkeit von H\textsubscript{2}-Infrastruktur hängt stark vom regionalen Kontext ab. Aktuelle Kostenanalysen zeigen für Wasserstoff-Tankstellen in HDV-Corridoren (Heavy-Duty Vehicles) LCOH-Werte (Levelized Cost of Hydrogen) zwischen €3,80 und €11,70/kg, abhängig von Stationskapazität (4--18 MTPD), Auslastung (30--80\%) und Versorgungsmethode \cite{Chung2024}. Detaillierte Kostenanalysen für verschiedene Stationstypen und -größen variieren zwischen €0,34 und €1,10/kg bei Vollauslastung, während Lieferkosten (inklusive Transport) €0,58 bis €4,92/kg betragen \cite{Eissler2023}. Diese Heterogenität erfordert regionsabhängige Optimierungsansätze.

% ============================================================
% PROBLEMSTELLUNG
% ============================================================

\section*{Problemstellung / Herausforderung und Forschungsstand}
\addcontentsline{toc}{section}{Problemstellung / Herausforderung und Forschungsstand}

\textbf{Das Kern-Dilemma (Chicken-and-Egg Problem):} Investoren wollen nur in H\textsubscript{2}-Infrastruktur investieren, wenn ausreichend Nachfrage und regulatorische Klarheit vorhanden sind. Andererseits können Unternehmen nicht in H\textsubscript{2}-basierte Produktionsprozesse umsteigen, wenn die Infrastruktur fehlt. Dieses Koordinationsproblem verzögert weltweit die Dekarbonisierung. Die beobachtete Diskrepanz zwischen Ankündigungen und Realisierung wird durch vier systematische Herausforderungen verstärkt:

\subsection*{Vier zentrale Barrieren}

\begin{enumerate}
  \item \textbf{Regulatorisch:} Unterschiedliche internationale Standards (ISO 14687, EU-Gasrichtlinie, nationale Sicherheitsvorschriften) erschweren grenzüberschreitende Projekte. Die ECH2A-Roadmap identifiziert Lücken in Standardisierung und regulatorischer Harmonisierung als kritische Implementierungshürden \cite{ECH2A2023}.

  \item \textbf{Technisch:} Wasserstoff-Speicherung und -Transport erfordern spezialisierte Infrastruktur. Optionen wie Salzkavernen-Speicherung, spezialisierte Pipeline-Beschichtungen und flüssiger Wasserstoff-Transport (LH\textsubscript{2}) mit energieintensiven Verdichtungsanlagen sind technisch gelöst, aber kostenintensiv. Alternativen wie Elektrolyseure mit Kompressoren (mechanisch oder elektrochemisch) bieten verschiedene CAPEX/OPEX-Tradeoffs \cite{Prokopou2025}.

  \item \textbf{Ökonomisch:} Grüner Wasserstoff ist aktuell 3--8 €/kg; Zielpreis bis 2030 <2 €/kg. Diese Kostenreduktion hängt von Skalierung, Strompreisen und Subventionen ab. Kostenoptimale Szenarien (z.B. Integration von PV-Anlagen bei Depot-Scale-Anwendungen) erfordern aktuell 58\% Subventionen für wirtschaftliche Rendite \cite{Vizza2025}. Wirtschaftlichkeit ist stark von regionaler Stromverfügbarkeit und -kosten abhängig.

  \item \textbf{Logistisch:} Wasserstoff ist dünn und schwer zu transportieren. Verschiedene Transport-Modi (Pipeline, Schiff, Truck, chemische Träger wie Ammoniak oder LOHC) haben unterschiedliche Kosten und Eignung je nach Distanz und Volumen. Supply-Chain-Integration zwischen Produktion und Nachfrage erfordert optimierte Netzwerk-Topologien \cite{Sujan2024}.
\end{enumerate}

\subsection*{Stand der Forschung}

Bestehende Arbeiten konzentrieren sich häufig auf Einzelaspekte (Technologie, Umwelt, einzelne Märkte). Ein systematischer Optimierungsrahmen für H\textsubscript{2}-Infrastruktur-Planung unter Unsicherheit wurde von \textcite{Efthymiadou2025} entwickelt, adressiert aber primär lokale Optimierungsprobleme. Es fehlt eine integrierte, \textbf{regionsübergreifende Analyse}, wie regulatorische, technische, ökonomische und logistische Dimensionen zusammenspielen und welche region-abhängigen Implementierungsstrategien unter realistischen Kosten- und Finanzierungsbedingungen am vielversprechendsten sind.

% ============================================================
% FORSCHUNGSFRAGEN
% ============================================================

\section*{Formulierung der Forschungsfragen}
\addcontentsline{toc}{section}{Formulierung der Forschungsfragen}

\textbf{Hauptfrage:}
\begin{quote}
  Welche Implementierungsstrategien für regionale H\textsubscript{2}-Infrastruktur sind unter gegebenen regulatorischen, technologischen und ökonomischen Rahmenbedingungen wirtschaftlich rentabel, und wie unterscheiden sie sich zwischen den Regionen Europa, Asien-Pazifik und Australien?
\end{quote}

\textbf{Unterfragen:}
\begin{enumerate}
  \item \textbf{RQ1:} Welche regulatorischen, technischen, wirtschaftlichen und logistischen Barrieren hemmen die H\textsubscript{2}-Infrastrukturentwicklung in unterschiedlichen Regionen, und wie werden sie in bestehenden Strategien adressiert?

  \item \textbf{RQ2:} Wie unterscheiden sich regionale Implementierungsmodelle (EU-Backbone-Modell mit zentraler Pipeline-Infrastruktur vs. asiatisches Import-Modell mit maritimem Transport vs. australisches Export-Modell mit lokaler Produktion) in ihrer technischen und ökonomischen Struktur?

  \item \textbf{RQ3:} Welche Implementierungsstrategien und Szenarien (zentral vs. dezentral, Pipeline vs. maritim, mit/ohne Subventionen) zeigen das beste Kosten-Nutzen-Verhältnis unter realistischen Kostenschätzungen und Finanzierungsbedingungen?
\end{enumerate}

% ============================================================
% ZIELSETZUNG
% ============================================================

\section*{Zielsetzung und Aufgabenstellung}
\addcontentsline{toc}{section}{Zielsetzung und Aufgabenstellung}

Die Bachelorarbeit verfolgt das Ziel:

\textbf{Empirische und quantitative Grundlagen schaffen} für die Priorisierung von H\textsubscript{2}-Infrastrukturinvestitionen durch systematische Benchmarking und ökonomische Bewertung:

\begin{itemize}
  \item Vergleichende Analyse von 3 regionalen Implementierungsmodellen (Europa, Asien-Pazifik, Australien) anhand von 4 Dimensionen (Regulierung, Technologie, Ökonomie, Logistik)
  \item Systematische Identifikation von Barrieren und deren Bedeutung in regionalen Strategien
  \item Quantitative Kostennutzenanalyse (CBA mit NPV-Berechnungen) für 2--3 realistische Szenarios pro Region
  \item Sensitivitätsanalyse zur Prüfung von politischen Hebeln (Subventionen, CO\textsubscript{2}-Preise)
\end{itemize}

\textbf{Die Arbeit wird:}
\begin{enumerate}
  \item \textbf{Theoretische Grundlagen} schaffen (H\textsubscript{2}-Technologie, Supply-Chain-Infrastruktur, regionale Strategien, Barrieren-Klassifikation)

  \item \textbf{Benchmarking durchführen:} Entwicklung und Anwendung einer 4D-Vergleichsmatrix (regulatorisch, technisch, ökonomisch, logistisch) auf 3 Regionen, basierend auf \textcite{Eissler2023}, \textcite{EUCommission2020}, und \textcite{METI2023}

  \item \textbf{Cost-Benefit-Analysis (CBA) anwenden:} NPV-Berechnung von 2--3 Implementierungsszenarien pro Region mit empirischen Kostendaten aus \textcite{Chung2024}, \textcite{Efthymiadou2025}, und \textcite{Sujan2024}

  \item \textbf{Sensitivitätsanalyse erstellen:} Analyse der Rentabilität unter Variation von Strompreisen, Subventionsniveaus und CO\textsubscript{2}-Preisen, informiert durch \textcite{Odenweller2025} und \textcite{Vizza2025}

  \item \textbf{Praktische Empfehlungen} formulieren für Entscheidungsträger (Politiker, Infrastrukturunternehmen, Energieunternehmen)
\end{enumerate}

% ============================================================
% GLIEDERUNG
% ============================================================

\section*{Gliederung}
\addcontentsline{toc}{section}{Gliederung}

\subsection*{1. Einleitung (4--5 Seiten)}
\begin{enumerate}
  \item[1.1] Motivation: Wasserstoff als Schlüsseltechnologie der Energiewende
  \item[1.2] Problemstellung: Das Chicken-and-Egg Dilemma und die Implementierungslücke \cite{Odenweller2025}
  \item[1.3] Forschungsfragen und Ziele
  \item[1.4] Aufbau der Arbeit
\end{enumerate}

\subsection*{2. Theoretische Grundlagen (8--10 Seiten)}
\begin{enumerate}
  \item[2.1] Wasserstoff-Technologie und Herstellung
  \begin{itemize}
    \item Produktionsmethoden (grauer, blauer, grüner Wasserstoff)
    \item Speicherung und Transport: Pipeline, flüssig (LH\textsubscript{2}), chemische Träger (Ammoniak, LOHC)
    \item Kostenstrukturen und Technologie-Roadmaps
  \end{itemize}

  \item[2.2] Supply-Chain- und Infrastrukturkonzepte
  \begin{itemize}
    \item Netzwerkdesign und räumliche Optimierung
    \item Interdependenzen zwischen Produktion, Speicherung, Transport und Nachfrage
    \item Dezentralisierung vs. zentrale Pipeline-Modelle
  \end{itemize}

  \item[2.3] Globale und regionale H\textsubscript{2}-Infrastrukturlandschaft
  \begin{itemize}
    \item Aktueller Stand: Investitionen, Projekte, regionale Strategien
    \item EU Hydrogen Backbone und europäische Roadmap \cite{EUCommission2020}
    \item Japan/Korea Import-Strategien und Südostasien-Integration \cite{METI2023}
    \item Australien Export-Potenzial und internationale Supply Chains
  \end{itemize}

  \item[2.4] Barrieren und Erfolgsfaktoren der Implementierung
  \begin{itemize}
    \item Regulatorische Hürden: Standards, Sicherheit, Harmonisierung \cite{ECH2A2023}
    \item Technische Anforderungen: Infrastruktur, Speicherung, Transport-Optionen \cite{Prokopou2025}
    \item Ökonomische Rentabilität: Kostenstrukturen, Break-Even-Punkte, Subventionsbedarf \cite{Vizza2025}
    \item Logistische Koordination und Netzwerk-Optimierung \cite{Sujan2024}
  \end{itemize}
\end{enumerate}

\subsection*{3. Methodik (3--4 Seiten)}
\begin{enumerate}
  \item[3.1] Forschungsdesign: Vergleichende Analyse mit Benchmarking und CBA
  \item[3.2] Benchmarking-Framework: 4 Dimensionen × 3 Regionen
  \begin{itemize}
    \item Dimension 1: Regulatorische Readiness
    \item Dimension 2: Technische Infrastruktur-Status
    \item Dimension 3: Kosten-Struktur und Geschäftsmodelle (basierend auf \cite{Eissler2023}, \cite{Chung2024})
    \item Dimension 4: Logistische und Netzwerk-Aspekte
  \end{itemize}
  \item[3.3] Cost-Benefit-Analysis Modell
  \begin{itemize}
    \item NPV-Rahmenwerk mit empirischen Kostenparametern
    \item Szenario-Definition (zentral vs. dezentral, Pipeline vs. Maritime)
    \item Diskontrate und Zeithorizont
  \end{itemize}
  \item[3.4] Sensitivitätsanalyse: Parameter und Szenarien
  \item[3.5] Datenquellen: Literatur-basierte Analyse mit empirischen Kostendaten
  \item[3.6] Geografischer und methodischer Scope
\end{enumerate}

\subsection*{4. Ergebnisse (12--15 Seiten)}
\begin{enumerate}
  \item[4.1] Benchmarking-Ergebnisse: Regionale Vergleichsmatrix
  \begin{itemize}
    \item Regulatorische Readiness pro Region
    \item Technische Infrastruktur-Status und Reifungsgrad
    \item Kosten-Struktur für verschiedene Stationstypen und Transportmodi
    \item Implementierungs-Zeithorizonte und -Barrieren
  \end{itemize}

  \item[4.2] Cost-Benefit-Analysis Ergebnisse
  \begin{itemize}
    \item Szenario-Definition: (zentralisiert + Pipeline) vs. (dezentralisiert + maritim) vs. (hybrid)
    \item NPV-Berechnungen für 2--3 Szenarien pro Region
    \item Break-Even-Analysen für verschiedene Auslastungsszenarien
    \item Kostenvergleiche und kritische Parameter
  \end{itemize}

  \item[4.3] Sensitivitätsanalyse
  \begin{itemize}
    \item Einfluss von Subventionsniveaus auf Rentabilität
    \item CO\textsubscript{2}-Preis-Variation und Carbon-Pricing-Szenarien
    \item Technologische Kostenreduktionen und Lernkurven
    \item Strompreis- und Wechselkurs-Szenarien
  \end{itemize}
\end{enumerate}

\subsection*{5. Diskussion (8--10 Seiten)}
\begin{enumerate}
  \item[5.1] Synthese der Benchmarking-Ergebnisse: Regionale Unterschiede und Muster
  \item[5.2] Interpretation der CBA-Ergebnisse: Welche Szenarien sind unter realistischen Bedingungen rentabel?
  \item[5.3] Beantwortung der Forschungsfragen
  \item[5.4] Regionale Implikationen und politische Trade-offs
  \item[5.5] Validierung gegenüber beobachteten Implementierungslücken \cite{Odenweller2025}
  \item[5.6] Limitationen und offene Fragen
\end{enumerate}

\subsection*{6. Fazit (3--4 Seiten)}
\begin{enumerate}
  \item[6.1] Zusammenfassung der Erkenntnisse
  \item[6.2] Handlungsempfehlungen für verschiedene Stakeholder
  \begin{itemize}
    \item Politische Entscheidungsträger
    \item Infrastrukturunternehmen
    \item Energieunternehmen und Industrieabnehmer
  \end{itemize}
  \item[6.3] Ausblick und zukünftige Forschung
\end{enumerate}

\newpage

% ============================================================
% LITERATURVERZEICHNIS
% ============================================================

\printbibliography[title=Literaturverzeichnis]

\newpage

% ============================================================
% CONCEPT MAP
% ============================================================

\section*{Concept Map}
\addcontentsline{toc}{section}{Concept Map}

\begin{center}
  \textbf{H\textsubscript{2}-Infrastruktur-Deployment in globalen Lieferketten: Analyserahmen}

  \vspace{1cm}

  \fbox{
    \begin{minipage}{0.9\textwidth}
      \centering
      \textbf{Energiewende \& Dekarbonisierung (Schwer-zu-Elektrifizieren Sektoren)} \\
      \vspace{0.1cm}
      \small \textit{(Motivation: EU Strategy 2020, METI 2023)}

      \vspace{0.8cm}
      $\downarrow$

      \vspace{0.8cm}

      \begin{tabular}{c c c}
        \fbox{\textbf{H\textsubscript{2}-Technologie}} & \fbox{\textbf{Globale Lücke}} & \fbox{\textbf{Regionale}} \\
        \fbox{\textbf{\& Kosten}} & \fbox{\textbf{(2025)}} & \fbox{\textbf{Strategien}} \\
        & \fbox{\textbf{}} & \\
        Produktion & 95\% Kapazität & EU Backbone \\
        Speicherung & nicht realisiert & Asien Import \\
        Transport & (Odenweller2025) & AU Export \\
        LCOH: €3.80--€11.70/kg & & \\
        (Chung2024, Eissler2023) & & \\
      \end{tabular}

      \vspace{0.8cm}
      $\downarrow$

      \vspace{0.8cm}

      \begin{tabular}{c c c}
        \fbox{\textbf{4 Barrieren}} & \fbox{\textbf{Optimierungs-}} & \fbox{\textbf{Implementierungs-}} \\
        \fbox{\textbf{}} & \fbox{\textbf{Methoden}} & \fbox{\textbf{Strategien}} \\
        & & \\
        Regulierung & Benchmarking & Dezentral + \\
        (ECH2A2023) & (4 Dimensionen) & Pipeline \\
        Technologie & & \\
        (Prokopou2025) & Cost-Benefit & Hybrid \\
        Ökonomie & Analysis (NPV) & (Sujan2024, \\
        (Vizza2025) & (Efthymiadou2025) & Eissler2023) \\
        Logistik & & \\
        (Sujan2024) & Sensitivitäts- & Maritim + \\
        & analyse & Dezentral \\
        & (Odenweller2025) & \\
      \end{tabular}

      \vspace{0.8cm}
      $\downarrow$

      \vspace{0.8cm}

      \fbox{\textbf{Handlungsempfehlungen: Region-spezifische Strategien mit Best-Cost-Benefit-Ratio}}

    \end{minipage}
  }
\end{center}

\vspace{1cm}

\noindent \textbf{Methodik-Integration:}
\begin{itemize}
  \item \textbf{Benchmarking:} Systematischer Vergleich der 4 Dimensionen über 3 Regionen mit empirischen Daten
  \item \textbf{Cost-Benefit Analysis:} Quantifiziert wirtschaftliche Rentabilität von Szenarien mit realistischen Kostendaten
  \item \textbf{Sensitivitätsanalyse:} Testet Robustheit gegenüber politischen und ökonomischen Parametern
  \item \textbf{Validierung:} Erklärt die beobachtete Implementierungslücke und informiert zukünftige Investitionen
\end{itemize}

\end{document}
