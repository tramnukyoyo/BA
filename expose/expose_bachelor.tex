\documentclass[12pt, a4paper, oneside]{article}
\usepackage[utf8]{inputenc}
\usepackage[ngerman]{babel}
\usepackage{csquotes}
\usepackage[hidelinks]{hyperref}
\usepackage{graphicx}
\usepackage[backend=biber,style=authoryear,sorting=nyt]{biblatex}
\usepackage{geometry}
\usepackage{setspace}
\usepackage{ragged2e}
\usepackage{fancyhdr}

% Margins
\geometry{
  top=2.5cm,
  bottom=2.5cm,
  left=2.5cm,
  right=2.5cm
}

% Bibliography
\addbibresource{references_h2_CORE.bib}
% Supplementary source: references_h2.bib (archived papers for background reference)

% Line spacing
\setstretch{1.5}

% Header/Footer
\pagestyle{fancy}
\fancyhead{}
\fancyfoot{}
\fancyfoot[C]{\thepage}
\renewcommand{\headrulewidth}{0pt}

% Title page setup
\title{
  \textbf{Wasserstofftankstellen in globalen Lieferketten}\\
  \vspace{0.5cm}
  \normalsize Implementierungsstrategien und Herausforderungen\\
  \vspace{0.3cm}
  \normalsize Exposé für eine Bachelorarbeit
}
\author{Denny}
\date{\today}

\begin{document}

\maketitle

\thispagestyle{empty}

\newpage

\tableofcontents
\thispagestyle{empty}

\newpage
\setcounter{page}{1}

% ============================================================
% AUSGANGSLAGE
% ============================================================

\section*{Ausgangslage}
\addcontentsline{toc}{section}{Ausgangslage}

Die Dekarbonisierung des Verkehrssektors erfordert alternative Antriebskonzepte. Wasserstoff (H\textsubscript{2}) ist eine vielversprechende Lösung für schwere Nutzfahrzeuge (HDV), da Batterie-Elektromobilität für diesen Sektor technisch und wirtschaftlich herausfordernd bleibt. Allerdings ist der Aufbau eines flächendeckenden Netzes von Wasserstofftankstellen (Hydrogen Refueling Stations, HRS) mit erheblichen Herausforderungen verbunden.

\subsection*{Die Implementierungslücke bei Wasserstofftankstellen (2025)}

Trotz wachsender politischer Unterstützung zeigt sich eine signifikante Lücke zwischen geplanter und realisierter HRS-Infrastruktur. Die globale H\textsubscript{2}-Infrastruktur weist insgesamt eine erhebliche Lücke zwischen Ankündigungen und Realisierung auf: Von den weltweit angekündigten H\textsubscript{2}-Kapazitäten im Jahr 2022 wurden nur 2\% pünktlich realisiert; 71\% verzeichneten Verzögerungen und 25\% wurden ganz aufgegeben \cite{Odenweller2025}. Diese Diskrepanz trifft besonders auf HRS-Netzwerke zu, wo Investoren garantierte Nachfrage benötigen, während Flottenbetreiber flächendeckende Infrastruktur voraussetzen.

Regionale HRS-Strategien unterscheiden sich erheblich \cite{Samsun2022, Kim2023}:

\begin{itemize}
  \item \textbf{Europäische Union:} Die EU-Verordnung für Alternative Fuels Infrastructure (AFIR) und die TEN-T-Richtlinie verpflichten Mitgliedstaaten, HRS-Netzwerke entlang von Verkehrskorridoren aufzubauen \cite{EUCommission2020}. Ziel: bis 2031 HRS alle 100 km entlang TEN-T-Hauptachsen (über 40.000 km Streckenlänge).

  \item \textbf{Japan:} Die nationale Wasserstoff-Strategie (2023) \cite{METI2023} priorisiert den Aufbau von 160-320 öffentlichen HRS bis 2030 für Brennstoff-Zell-Fahrzeuge (FCEV), mit Fokus auf Stadtzentren und Verkehrskorridore.

  \item \textbf{Australien:} Mit reichlich vorhandener erneuerbarer Energie und geografischer Nähe zu asiatischen Märkten entwickelt sich eine HRS-Industrie als Drehscheibe für H\textsubscript{2}-Versorgung zu Japan und Korea. Die regionale Vergleichsanalyse zeigt unterschiedliche Implementierungsansätze zwischen europäischen Korridoren und asiatischen Nachfrage-fokussierten Modellen \cite{Kim2023}.
\end{itemize}

\subsection*{Kostenstruktur und Wirtschaftlichkeit von HRS}

Die wirtschaftliche Rentabilität von Wasserstofftankstellen hängt stark von Standort, Stationsgröße und Auslastung ab. Kostenanalysen zeigen für 35-MPa-HRS (heute Standard) Investitionskosten (CAPEX) von €500k bis €2M pro Station, abhängig von regionalen Faktoren und Technik. Die Betriebskosten (OPEX) betragen €50k-€200k/Jahr, inklusive Wartung, Personal und Wasserstoff-Lieferkosten \cite{Wu2024}. Wasserstoff-Kosten liegen derzeit bei €7-12/kg grün (abhängig von Stromkosten), müssen aber bis 2030 auf €2-3/kg sinken für Rentabilität ohne Subventionen \cite{Chung2024, Eissler2023}. Bei typischen Auslastungsszenarien (30--80\%) zeigen ökonomische Modelle, dass 58\% Subventionen erforderlich sind für Break-Even-Profitabilität \cite{Vizza2025}. Eine Detailanalyse von HRS-Betriebsmodi unter verschiedenen Szenarien verstärkt diese Erkenntnisse \cite{Wu2024}. Diese Variation erfordert regionsabhängige Strategien und differenzierte Finanzierungsmechanismen.

% ============================================================
% PROBLEMSTELLUNG
% ============================================================

\section*{Problemstellung / Herausforderung und Forschungsstand}
\addcontentsline{toc}{section}{Problemstellung / Herausforderung und Forschungsstand}

\textbf{Das Kern-Dilemma (Chicken-and-Egg Problem):} Investoren wollen nur in HRS investieren, wenn ausreichend Nachfrage (verfügbare Wasserstoff-Fahrzeugflotten) und regulatorische Klarheit vorhanden sind. Andererseits können Flottenbetreiber nicht in Wasserstoff-Fahrzeuge investieren, wenn flächendeckende HRS-Netzwerke fehlen. Dieses Koordinationsproblem verzögert weltweit die Dekarbonisierung des Verkehrssektors. Die beobachtete Diskrepanz zwischen angekündigten HRS-Projekten und ihrer tatsächlichen Realisierung wird durch drei systematische Barrierekategorien verstärkt:

\subsection*{Drei zentrale Barriere-Kategorien}

\begin{enumerate}
  \item \textbf{Regulatorische Hürden:} Standards und Zertifizierung sind HRS-spezifisch fragmentiert. Unterschiedliche Drücke (35 MPa vs. 70 MPa), Sicherheitsstandards (ISO 14687, EU-Richtlinien, nationale Vorschriften) und Zertifizierungsprozesse erschweren interoperabilität und grenzüberschreitende Projekte \cite{Genovese2023}. Die ECH2A-Roadmap identifiziert Harmonisierungslücken als kritische Implementierungshürden \cite{ECH2A2023}. Zusätzlich regeln EU-Verordnungen (AFIR, Richtlinie 2014/94/EU) HRS-Netzwerk-Aufbau und schaffen neue Compliance-Anforderungen \cite{EUCommission2020}.

  \item \textbf{Technische Hürden:} HRS erfordern zuverlässige Kompressoren, Speichertanks, Kühlanlage und Dispensing-Systeme. Technische Herausforderungen umfassen: (a) Zuverlässigkeit und Verfügbarkeit (HRS sollten >95\% Uptime erreichen), (b) Wartungsanforderungen und Ausfallsicherheit, (c) Speicher- und Transportlogistik (Trailer, Pipeline, On-site Elektrolyse), (d) Skalierbarkeit je nach Standort und Nachfrage \cite{Genovese2024, Eissler2023, Prokopou2025}.

  \item \textbf{Logistische Hürden:} Die Versorgung von HRS mit Wasserstoff ist komplex und ortsabhängig. Optionen sind: (a) Central Production + Trailer-Transport (flexibel, skalierbar, aber teuer), (b) Pipeline-Versorgung (kostengünstig bei hohem Volumen, aber inflexibel), (c) On-site Elektrolyse (dezentral, abhängig von Stromverfügbarkeit). Optimale Netzwerk-Topologie hängt von Standort, Nachfrage und Wasserstoff-Quellen ab. Supply-Chain-Integration von Produktion über Transport zu HRS erfordert koordinierte Planung \cite{Raeesi2024, Sujan2024, Eissler2023}.
\end{enumerate}

\subsection*{Stand der Forschung}

Bestehende Arbeiten zu HRS konzentrieren sich häufig auf Einzelaspekte: Technische Designs und Zuverlässigkeit \cite{Prokopou2025}, ökonomische Szenarien \cite{Chung2024, Vizza2025}, oder regionale Markt-Analysen \cite{METI2023, EUCommission2020}. Allerdings fehlt eine integrierte, \textbf{systematische Analyse}, die folgende Fragen beantwortet: (1) Wie wirken regulatorische, technische, logistische und ökonomische Barrieren zusammen? (2) Welche HRS-Integrations-Szenarien (Korridore, urbane Cluster, Industrie-Hubs) sind unter regionalen Bedingungen optimal? (3) Welche Implementierungsstrategien unter realistischen Kosten und Finanzierungsbedingungen in verschiedenen Regionen wirtschaftlich tragfähig? Diese Arbeit schließt diese Forschungslücke durch einen integrativen Ansatz der HRS-Implementierung.

% ============================================================
% FORSCHUNGSFRAGEN
% ============================================================

\section*{Formulierung der Forschungsfragen}
\addcontentsline{toc}{section}{Formulierung der Forschungsfragen}

\textbf{Hauptforschungsfrage:}
\begin{quote}
  Welche Implementierungsstrategien für Wasserstofftankstellen (HRS) sind unter gegebenen regulatorischen, logistischen und technischen Rahmenbedingungen wirtschaftlich tragfähig, und wie unterscheiden sie sich zwischen Regionen (Europa, Asien-Pazifik, Australien)?
\end{quote}

\textbf{Teilforschungsfragen:}
\begin{enumerate}
  \item \textbf{RQ1:} Welche regulatorischen, logistischen und technischen Barrieren behindern die HRS-Implementierung in Europa, Asien und Australien, und wie adressieren regionale Strategien diese unterschiedlich?

  \item \textbf{RQ2:} Welche HRS-Integrationsszenarien (Korridor-basiert, urbane Cluster, Industrie-Hubs) sind unter verschiedenen H\textsubscript{2}-Versorgungslogistiken (Trailer, Pipeline, On-site Elektrolyse) optimal?

  \item \textbf{RQ3:} Unter welchen Bedingungen (Subventionsniveaus, CO\textsubscript{2}-Preisen, Nachfrageszenarien) erreichen HRS-Implementierungsstrategien wirtschaftliche Rentabilität in verschiedenen Regionen?
\end{enumerate}

% ============================================================
% ZIELSETZUNG
% ============================================================

\section*{Zielsetzung und Aufgabenstellung}
\addcontentsline{toc}{section}{Zielsetzung und Aufgabenstellung}

Die Bachelorarbeit verfolgt das Ziel:

\textbf{Empirische und quantitative Grundlagen schaffen} für die Priorisierung von HRS-Investitionen durch systematische Barrierenanalyse und ökonomische Bewertung:

\begin{itemize}
  \item Vergleichende Analyse von 3 regionalen HRS-Implementierungsmodellen (Europa mit Korridoransatz, Asien mit Nachfrage-fokussiertem Ansatz, Australien mit Export-Hub-Modell) entlang von 3 Barriere-Dimensionen (Regulatorisch, Logistisch, Technisch)
  \item Szenarioentwicklung: Definition und Bewertung von HRS-Integrationsszenarien (Korridor-basiert, urbane Cluster, Industrie-Hubs) unter verschiedenen Versorgungslogistiken
  \item Quantitative Kostennutzenanalyse (Cost-Benefit-Analysis mit NPV und ROI) für 2--3 realistische Szenarien pro Region basierend auf empirischen Kostendaten
  \item Sensitivitätsanalyse zur Prüfung kritischer Parameter (Subventionsniveaus, CO\textsubscript{2}-Preisierung, Stromkosten, Nachfrageentwicklung)
\end{itemize}

\textbf{Die Arbeit wird:}
\begin{enumerate}
  \item \textbf{HRS-spezifische Grundlagen} schaffen (Wasserstoff-Refueltechnologie, HRS-Komponenten, Versorgungslogistik, regionale Implementierungsmodelle, Barrieren-Klassifikation)

  \item \textbf{Barrieren-Benchmarking durchführen:} Systematische Vergleichsmatrix (3 Barriere-Dimensionen × 3 Regionen) mit empirischen Beispielen und Policy-Ansätzen, basierend auf aktuellen regionalen Strategien und Literatur

  \item \textbf{Szenarioentwicklung durchführen:} Definition von 3 HRS-Integrationsszenarien (Korridor, Cluster, Hub) mit Netzwerk-Optimierungsprinzipien und räumlicher Planung

  \item \textbf{Cost-Benefit-Analysis (CBA) anwenden:} NPV- und ROI-Berechnungen für 2--3 realistische Szenarien pro Region mit disaggregierten Kostendaten (CAPEX, OPEX, H\textsubscript{2}-Kosten, Transport)

  \item \textbf{Sensitivitätsanalyse erstellen:} Analyse der Rentabilität unter Variation kritischer Parameter (Subventionen 0\%-80\%, CO\textsubscript{2}-Preise, Stromkosten, Auslastung 30\%-80\%)

  \item \textbf{Praktische Handlungsempfehlungen} formulieren für Entscheidungsträger (Policymaker, Infrastrukturbetreiber, Flottenbetreiber) pro Region
\end{enumerate}

% ============================================================
% GLIEDERUNG
% ============================================================

\section*{Gliederung}
\addcontentsline{toc}{section}{Gliederung}

\subsection*{1. Einleitung (4--5 Seiten)}
\begin{enumerate}
  \item[1.1] Motivation: Wasserstoff als Schlüsseltechnologie der Energiewende
  \item[1.2] Problemstellung: Das Chicken-and-Egg Dilemma und die Implementierungslücke \cite{Odenweller2025}
  \item[1.3] Forschungsfragen und Ziele
  \item[1.4] Aufbau der Arbeit
\end{enumerate}

\subsection*{2. Theoretische Grundlagen (8--10 Seiten)}
\begin{enumerate}
  \item[2.1] Wasserstoff-Refuelling-Technologie
  \begin{itemize}
    \item HRS-Komponenten: Kompressoren, Speichertanks, Dispenser, Kühlanlage
    \item Druckstandards (35 MPa vs. 70 MPa) und deren Auswirkungen auf Kosten und Reichweite
    \item Versorgungsoptionen: Central Production, Pipeline, On-site Elektrolyse
    \item Zuverlässigkeitsanforderungen und Verfügbarkeitsziele
  \end{itemize}

  \item[2.2] HRS-Versorgungskette und Netzwerkdesign
  \begin{itemize}
    \item Supply-Chain-Integration von H\textsubscript{2}-Produktion zu Endnutzer \cite{Raeesi2024}
    \item Transportmodi: Trailer, Pipeline, Schiff (für internationale Supply Chains)
    \item Netzwerk-Topologie: Dezentral (Cluster), zentral (Backbone), hybrid (Corridor)
    \item Räumliche Optimierung und Standortwahl für HRS \cite{Isaac2023}
  \end{itemize}

  \item[2.3] Globale und regionale HRS-Landschaft
  \begin{itemize}
    \item Aktueller Deployment-Status: Anzahl Stationen, Projekte, Timelines
    \item EU-Strategie: AFIR-Verordnung, TEN-T-Korridore, Zielwerte bis 2031
    \item Asien-Pazifik: Japan FCE-Strategie, Korea, Australien als Exporteur
    \item Regionale Unterschiede in Geschäftsmodellen und Finanzierungsansätzen
  \end{itemize}

  \item[2.4] Regulatorische, logistische und technische Hürden bei HRS
  \begin{itemize}
    \item Regulatorische Hürden: ISO-Standards, nationale Zertifizierung, Druckstandards-Harmonisierung, AFIR-Compliance
    \item Technische Hürden: Zuverlässigkeit, Wartung, Skalierbarkeit, Speicher- und Transportlogistik
    \item Logistische Hürden: H\textsubscript{2}-Transportkosten, Versorgungskettenintegration, Lageroptimierung
    \item Ökonomische Barrieren: Hohe Kapitalkosten, geringe Auslastung, Subventionsabhängigkeit
  \end{itemize}
\end{enumerate}

\subsection*{3. Methodik (3--4 Seiten)}
\begin{enumerate}
  \item[3.1] Forschungsdesign: Vergleichende Analyse mit Barrieren-Benchmarking und CBA
  \item[3.2] Barrieren-Benchmarking: 3 Dimensionen × 3 Regionen
  \begin{itemize}
    \item Dimension 1: Regulatorische Hürden (Standards, Zertifizierung, nationale Regelwerke) \cite{Genovese2023}
    \item Dimension 2: Logistische Hürden (Versorgungskette, Transport, Speicherung) \cite{Raeesi2024}
    \item Dimension 3: Technische Hürden (Zuverlässigkeit, Wartung, Skalierung) \cite{Genovese2024}
    \item Anwendung auf 3 Regionen: Europa, Asien-Pazifik, Australien \cite{Kim2023, Samsun2022}
  \end{itemize}
  \item[3.3] Szenarioentwicklung: HRS-Integrationsszenarien
  \begin{itemize}
    \item Szenario A: Korridor-basiert (Highway-Deployment, TEN-T-Modell) \cite{Kim2020, Isaac2023}
    \item Szenario B: Urbane Cluster (Stadtnahe HRS-Netzwerke) \cite{Isaac2023}
    \item Szenario C: Industrie-Hubs (Hafen- und Logistikzentren)
    \item Versorgungslogistik-Varianten pro Szenario \cite{Raeesi2024}
  \end{itemize}
  \item[3.4] Cost-Benefit-Analysis (CBA) Modell
  \begin{itemize}
    \item NPV-Berechnung mit 10-20 Jahre Zeithorizont
    \item Kostendisaggregation: CAPEX (Stationen, Speicherung), OPEX (Wartung, Personal), H\textsubscript{2}-Kosten, Transport \cite{Wu2024, Atabay2024}
    \item Empirische Kostenparameter aus Literatur
    \item Szenario-spezifische NPV-Berechnung pro Region
  \end{itemize}
  \item[3.5] Sensitivitätsanalyse
  \begin{itemize}
    \item Parameter: Subventionsniveaus (0\%-80\%), CO\textsubscript{2}-Preise (€0-€150/t), Stromkosten, Auslastung (30\%-80\%)
    \item Identifikation kritischer Erfolgsfaktoren für Rentabilität
    \item Szenarien-Vergleich unter Unsicherheit
  \end{itemize}
  \item[3.6] Datenquellen und Scope
  \begin{itemize}
    \item Primär: Literatur-basierte Analyse mit empirischen Kostendaten aus 10+ Core Papers
    \item Regionale Fokus: EU (AFIR/TEN-T), Japan (METI 2023), Australien (H\textsubscript{2})
    \item Methodischer Scope: Qualitative Barrierenanalyse + quantitative CBA
  \end{itemize}
\end{enumerate}

\subsection*{4. Ergebnisse (12--15 Seiten)}
\begin{enumerate}
  \item[4.1] Barrieren-Benchmarking-Ergebnisse: 3 × 3 Vergleichsmatrix
  \begin{itemize}
    \item Regulatorische Hürden pro Region (Standards, Zertifizierung, nationale Anforderungen)
    \item Logistische Hürden pro Region (Versorgungskosten, Transportmethoden, Verfügbarkeit)
    \item Technische Hürden pro Region (Zuverlässigkeit, Technologie-Status, Wartungsanforderungen)
    \item Gegenwärtige Implementierungsstrategien und deren Effektivität pro Region
  \end{itemize}

  \item[4.2] HRS-Integrationsszenarien und Netzwerk-Analyse
  \begin{itemize}
    \item Szenario A (Korridor): Deployment entlang Hauptverkehrsachsen, Kapazitäten, Kosten
    \item Szenario B (Cluster): Urbane Netzwerke, Abdeckung, Nachfrage-Profile
    \item Szenario C (Hubs): Industrie-Standorte, Großmengen-Szenarien
    \item Vergleich der Szenarien nach Flächenabdeckung, Investmentbedarf, Zeitrahmen
  \end{itemize}

  \item[4.3] Cost-Benefit-Analysis Ergebnisse
  \begin{itemize}
    \item NPV-Berechnungen für 2--3 Szenarien pro Region (20 Jahre Zeithorizont)
    \item CAPEX- und OPEX-Breakdowns für verschiedene HRS-Größen und -Typen
    \item Break-Even-Analysen: Bei welcher Auslastung wird Rentabilität erreicht?
    \item Szenario-Ranking nach wirtschaftlicher Attraktivität pro Region
  \end{itemize}

  \item[4.4] Sensitivitätsanalyse
  \begin{itemize}
    \item Einfluss Subventionen: NPV-Entwicklung von 0\% bis 80\% Subventionierung
    \item CO\textsubscript{2}-Preis-Szenarios: €0/t bis €150/t und deren Effekt auf Rentabilität
    \item Stromkosten-Variation: Auswirkungen auf Wasserstoff-Kosten und Gesamtrentabilität
    \item Auslastungs-Szenarien: 30\%, 50\%, 80\% und Break-Even-Punkte
  \end{itemize}
\end{enumerate}

\subsection*{5. Diskussion (8--10 Seiten)}
\begin{enumerate}
  \item[5.1] Synthese der Barrieren-Analyse: Regionale Muster und Unterschiede
  \begin{itemize}
    \item Vergleich: Welche Barrieren sind in welcher Region dominant?
    \item Effektivität regionaler Strategien zur Überwindung dieser Barrieren
  \end{itemize}

  \item[5.2] Interpretation der Szenario- und CBA-Ergebnisse
  \begin{itemize}
    \item Welche HRS-Integrationsszenarien sind unter realistischen Bedingungen rentabel?
    \item Break-Even-Punkte und kritische Erfolgsfaktoren pro Region
  \end{itemize}

  \item[5.3] Beantwortung der Forschungsfragen
  \begin{itemize}
    \item RQ1: Barrieren und ihre regionalen Unterschiede - Antwort und Implikationen
    \item RQ2: Optimale Szenarien für verschiedene Logistiken - Erkenntnisse
    \item RQ3: Wirtschaftlichkeit unter verschiedenen Policy-Szenarien - Ergebnisse
  \end{itemize}

  \item[5.4] Implikationen für verschiedene Stakeholder
  \begin{itemize}
    \item Policymaker: Welche Instrumente (Subventionen, Mandate, Standards) sind effektiv?
    \item Infrastrukturbetreiber: Welche Geschäftsmodelle sind nachhaltig?
    \item Flottenbetreiber: Welche regionalen Strategien reducieren Nachfrage-Risiko?
  \end{itemize}

  \item[5.5] Validierung gegen beobachtete Implementierungslücken
  \begin{itemize}
    \item Erklären die Ergebnisse, warum geplante Projekte verzögert/aufgegeben werden?
    \item Welche Policy-Änderungen könnten die Lücke schließen?
  \end{itemize}

  \item[5.6] Limitationen und offene Fragen
  \begin{itemize}
    \item Datenunfänzlichkeiten, Annahmen in der CBA-Modellierung
    \item Zukünftige Technologie-Entwicklungen (z.B. bessere Speicherung)
    \item Nachfrageprognosen für H\textsubscript{2}-Fahrzeuge
  \end{itemize}
\end{enumerate}

\subsection*{6. Fazit (3--4 Seiten)}
\begin{enumerate}
  \item[6.1] Zusammenfassung der Erkenntnisse
  \begin{itemize}
    \item Hauptergebnisse zu HRS-Barrieren pro Region
    \item Effektivste Integrationsszenarien und Bedingungen für Rentabilität
  \end{itemize}

  \item[6.2] Handlungsempfehlungen für verschiedene Stakeholder (regionsabhängig)
  \begin{itemize}
    \item \textbf{Policymaker:} Spezifische Policy-Instrumente (Subventionen, Standards, Mandate) für jede Region
    \item \textbf{Infrastrukturbetreiber:} Geschäftsmodelle, Finanzierungsquellen, Standortwahl-Strategien
    \item \textbf{Flottenbetreiber:} Nachfrage-Signale, Investitions-Timing, regionale Priorities
    \item \textbf{Investoren:} Rentabilitäts-Szenarien, Risikofaktoren, Opportunitäten pro Region
  \end{itemize}

  \item[6.3] Ausblick und zukünftiger Forschungsbedarf
  \begin{itemize}
    \item Technologie-Roadmaps: Wie sinken HRS-Kosten bis 2035?
    \item Nachfrageentwicklung: Welche FCEV-Adoption-Raten sind realistisch?
    \item Institutionelle Reformen: Wie können regulatorische Barrieren schneller gelöst werden?
    \item Globale Wasserstoff-Märkte: Wie beeinflussen Importmodelle die HRS-Entwicklung?
  \end{itemize}
\end{enumerate}

\newpage

% ============================================================
% LITERATURVERZEICHNIS
% ============================================================

\printbibliography[title=Literaturverzeichnis]

\newpage

% ============================================================
% CONCEPT MAP
% ============================================================

\section*{Konzeptioneller Rahmen}
\addcontentsline{toc}{section}{Konzeptioneller Rahmen}

Die Arbeit strukturiert sich entlang von drei Dimensionen der HRS-Implementierung:

\subsection*{1. Herausforderungen (Challenges)}

Die Implementierung von Wasserstofftankstellen steht vor drei Haupthürden:

\begin{itemize}
\item \textbf{Regulatorische Hürden:} Fragmentierung von Standards (ISO, nationale Vorschriften), Zertifizierungsprozesse, und grenzüberschreitende Regelungslücken

\item \textbf{Logistische Hürden:} H\textsubscript{2}-Transportmodi (Pipeline, Trailer, On-site Produktion), Versorgungskettenintegration von Produktion zu HRS, und Speicher-/Dispensing-Technologien

\item \textbf{Technische Hürden:} Zuverlässigkeit und Wartungsanforderungen, Kompressor- und Kühltechnologie, sowie Skalierbarkeit und Modularität
\end{itemize}

\subsection*{2. Analyse-Framework (Analysis Framework)}

Zur Analyse dieser Herausforderungen nutzt die Arbeit drei komplementäre Methoden:

\begin{itemize}
\item \textbf{Räumliche Planung \& Optimierung:} Standortwahl, Netzwerk-Topologie und Deployment-Szenarien für unterschiedliche geografische Kontexte

\item \textbf{Wirtschaftliche Bewertung:} Kostenstruktur-Analyse (CAPEX/OPEX), Net Present Value (NPV), Return on Investment (ROI), sowie Sensitivitätsanalyse für verschiedene Finanzierungsszenarien

\item \textbf{Globaler Marktkontext:} Regionale Deployment-Trends, strategische Unterschiede zwischen Europa, Asien-Pazifik und Australien, sowie Markt-Treiber und Adoption-Barrieren
\end{itemize}

\subsection*{3. Integrationsszenarien (Integration Scenarios)}

Die Thesis untersucht drei Hauptszenarien für die Integration von HRS in bestehende Infrastrukturen:

\begin{itemize}
\item \textbf{Korridor-basiert:} HRS entlang von Hauptverkehrsachsen (z.B. TEN-T-Netzwerk in der EU), mit Fokus auf Long-Haul-Transport

\item \textbf{Urbane Cluster:} Stadtnahe HRS-Netzwerke für städtische Mobilität (Busse, Taxis, Flotten), mit kleineren Stationen und höherer Nutzungsfrequenz

\item \textbf{Industrie-Hubs:} Hafen- und Logistikzentren mit Fokus auf Schwertransport und lokale Versorgungsketten
\end{itemize}

\vspace{0.5cm}

\noindent\textbf{Synthese:} Aus der Analyse dieser drei Dimensionen und Szenarien werden regionsabhängige Strategien und Handlungsempfehlungen abgeleitet, die Investoren, Policymaker und Infrastrukturbetreiber bei der effizienten HRS-Implementierung unterstützen.

\end{document}
