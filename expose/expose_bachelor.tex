\documentclass[12pt, a4paper, oneside]{article}
\usepackage[utf8]{inputenc}
\usepackage[ngerman]{babel}
\usepackage{csquotes}
\usepackage[hidelinks]{hyperref}
\usepackage{graphicx}
\usepackage[backend=biber,style=authoryear,sorting=nyt]{biblatex}
\usepackage{geometry}
\usepackage{setspace}
\usepackage{ragged2e}
\usepackage{fancyhdr}

% Margins
\geometry{
  top=2.5cm,
  bottom=2.5cm,
  left=2.5cm,
  right=2.5cm
}

% Bibliography
\addbibresource{references_h2.bib}

% Line spacing
\setstretch{1.5}

% Header/Footer
\pagestyle{fancy}
\fancyhead{}
\fancyfoot{}
\fancyfoot[C]{\thepage}
\renewcommand{\headrulewidth}{0pt}

% Title page setup
\title{
  \textbf{Wasserstoffinfrastruktur in globalen Lieferketten}\\
  \vspace{0.5cm}
  \normalsize Exposé für eine Bachelorarbeit
}
\author{Denny}
\date{\today}

\begin{document}

\maketitle

\thispagestyle{empty}

\newpage

\tableofcontents
\thispagestyle{empty}

\newpage
\setcounter{page}{1}

% ============================================================
% AUSGANGSLAGE
% ============================================================

\section*{Ausgangslage}
\addcontentsline{toc}{section}{Ausgangslage}

Die globale Energiewende erfordert die schnelle Dekarbonisierung von schwer zu elektrifizierenden Sektoren wie Stahlproduktion, Chemie und Langstreckentransport. Wasserstoff (H\textsubscript{2}) gilt dabei als Schlüsseltechnologie: Er kann als Energieträger für diese Sektoren dienen und in erneuerbaren Energiesystemen überschüssige Elektrizität speichern.

\subsection*{Der aktuelle Stand (2025)}

\begin{itemize}
  \item \textbf{Angebotslücke:} Die globale grüne H\textsubscript{2}-Produktion liegt bei etwa 5 Megatonnen pro Jahr, während die erwartete Nachfrage bis 2050 auf 500+ Megatonnen projiziert wird.
  \item \textbf{Infrastrukturdefizit:} Die meisten Länder verfügen nicht über spezialisierte H\textsubscript{2}-Transportnetze. Es gibt weltweit nur etwa 4.600 km Wasserstoff-Pipelines, während für die Skalierung hunderttausende Kilometer notwendig sind.
  \item \textbf{Regionale Unterschiede:} Industrieländer wie Deutschland und Japan investieren aktiv in H\textsubscript{2}-Infrastruktur, während andere Regionen aufgrund fehlender Finanzierung, Regulierung und technologischer Reife zurückbleiben.
\end{itemize}

\subsection*{Beispiele bestehender Initiativen}

\begin{itemize}
  \item \textbf{EU Hydrogen Backbone:} Geplantes europäisches Verbundnetz von 39.700 km Wasserstoff-Pipelines (2030--2040)
  \item \textbf{Japan/Südkorea:} Importorientierte Strategie mit Partnerschaften zu Australien und dem Nahen Osten
  \item \textbf{Australien:} Export-fokussiert, verfügt über große Mengen erneuerbarer Energie für H\textsubscript{2}-Produktion
\end{itemize}

Diese regionalen Unterschiede zeigen: Es gibt nicht einen einzigen "`richtigen Weg"', sondern unterschiedliche Implementierungsstrategien je nach geografischem Kontext und vorhandenen Ressourcen.

% ============================================================
% PROBLEMSTELLUNG
% ============================================================

\section*{Problemstellung / Herausforderung und Forschungsstand}
\addcontentsline{toc}{section}{Problemstellung / Herausforderung und Forschungsstand}

\textbf{Das Kern-Dilemma (Chicken-and-Egg Problem):} Investoren wollen nur in H\textsubscript{2}-Infrastruktur investieren, wenn ausreichend Nachfrage vorhanden ist. Andererseits können Unternehmen nicht in H\textsubscript{2}-basierte Produktionsprozesse umsteigen, wenn die Infrastruktur fehlt. Dieses Koordinationsproblem verzögert weltweit die Dekarbonisierung.

\subsection*{Aktuelle Herausforderungen}

\begin{enumerate}
  \item \textbf{Regulatorisch:} Unterschiedliche internationale Standards (ISO 14687, EU-Gasrichtlinie, nationale Sicherheitsvorschriften) erschweren grenzüberschreitende Projekte.

  \item \textbf{Technisch:} Wasserstoff-Speicherung und -Transport erfordern spezialisierte Infrastruktur (Salzkavernen für Speicherung, Rohre mit H\textsubscript{2}-Sperrschicht, spezialisierte Schiffe für Flüssigtransport).

  \item \textbf{Ökonomisch:} Grüner Wasserstoff ist noch teuer (aktuell 3--8 €/kg, Ziel bis 2030: <2 €/kg). Große Infrastrukturinvestitionen erfordern klare Geschäftsmodelle und staatliche Unterstützung.

  \item \textbf{Logistisch:} Wasserstoff ist dünn und schwer zu transportieren. Verschiedene Transport-Modi (Pipeline, Schiff, Truck, chemische Träger wie Ammoniak) haben unterschiedliche Kosten und Eignung je nach Distanz und Volumen.
\end{enumerate}

\subsection*{Stand der Forschung}

Bestehende Arbeiten konzentrieren sich oft auf Einzelaspekte (Technologie, Umwelt, Märkte). Es fehlt eine systematische Analyse, wie diese Dimensionen zusammenspielen und welche \textbf{regionsabhängigen Implementierungsstrategien} am vielversprechendsten sind.

% ============================================================
% FORSCHUNGSFRAGEN
% ============================================================

\section*{Formulierung der Forschungsfragen}
\addcontentsline{toc}{section}{Formulierung der Forschungsfragen}

\textbf{Hauptfrage:}
\begin{quote}
  Welche Implementierungsstrategien für Wasserstoffinfrastruktur sind in unterschiedlichen regionalen Kontexten wirtschaftlich rentabel und regulatorisch machbar?
\end{quote}

\textbf{Unterfragen:}
\begin{enumerate}
  \item \textbf{RQ1:} Welche regulatorischen, technischen, wirtschaftlichen und logistischen Barrieren hemmen die H\textsubscript{2}-Infrastrukturentwicklung in unterschiedlichen Regionen?

  \item \textbf{RQ2:} Wie unterscheiden sich regionale Ansätze (EU-Backbone-Modell vs. asiatisches Import-Modell vs. australisches Export-Modell) in der Überwindung dieser Barrieren?

  \item \textbf{RQ3:} Welche Implementierungsstrategien zeigen das beste Kosten-Nutzen-Verhältnis unter Berücksichtigung regionaler Rahmenbedingungen?
\end{enumerate}

% ============================================================
% ZIELSETZUNG
% ============================================================

\section*{Zielsetzung und Aufgabenstellung}
\addcontentsline{toc}{section}{Zielsetzung und Aufgabenstellung}

Die Abschlussarbeit verfolgt folgendes Ziel:

\textbf{Empirische Grundlagen schaffen} für die Priorisierung von H\textsubscript{2}-Infrastrukturinvestitionen durch:
\begin{itemize}
  \item Vergleichende Analyse von 3 regionalen Modellen (Europa, Asien-Pazifik, Australien)
  \item Systematische Bewertung von Barrieren und Erfolgsfaktoren
  \item Quantitative Kostennutzenanalyse für unterschiedliche Szenarios
\end{itemize}

\textbf{Die Arbeit wird:}
\begin{enumerate}
  \item \textbf{Theoretische Grundlagen} liefern (H\textsubscript{2}-Technologie, Supply-Chain-Konzepte, Barrieren)
  \item \textbf{Benchmarking durchführen:} Vergleichende Matrix der regionalen Ansätze (Regulierung, Technologie, Wirtschaft, Logistik)
  \item \textbf{Kostennutzenanalyse (CBA) anwenden:} NPV-Berechnung für 2--3 Implementierungsszenarien pro Region
  \item \textbf{Sensitivitätsanalyse erstellen:} Wie beeinflussen politische Variablen (Subventionen, CO\textsubscript{2}-Preise) die Rentabilität?
  \item \textbf{Praktische Empfehlungen} formulieren für Entscheidungsträger (Politiker, Infrastrukturunternehmen, Energieunternehmen)
\end{enumerate}

% ============================================================
% GLIEDERUNG
% ============================================================

\section*{Gliederung}
\addcontentsline{toc}{section}{Gliederung}

\subsection*{1. Einleitung (4--5 Seiten)}
\begin{enumerate}
  \item[1.1] Motivation: Wasserstoff als Schlüssel der Energiewende
  \item[1.2] Problemstellung: Das Chicken-and-Egg Dilemma
  \item[1.3] Forschungsfragen und Ziele
  \item[1.4] Aufbau der Arbeit
\end{enumerate}

\subsection*{2. Theoretische Grundlagen (8--10 Seiten)}
\begin{enumerate}
  \item[2.1] Wasserstoff-Technologie und Herstellung
  \begin{itemize}
    \item Produktionsmethoden (grauer, blauer, grüner Wasserstoff)
    \item Speicherung und Transport (Pipeline, flüssig, chemische Träger)
  \end{itemize}
  \item[2.2] Supply-Chain- und Infrastrukturkonzepte
  \begin{itemize}
    \item Netzwerkdesign und Interdependenzen
    \item Infrastrukturelles Bottleneck-Denken
  \end{itemize}
  \item[2.3] Globale H\textsubscript{2}-Infrastrukturlandschaft
  \begin{itemize}
    \item Aktueller Stand: Produktion, Projekte, regionale Strategien
    \item EU Hydrogen Backbone, Japan/Korea Import-Initiativen, Australien Export
  \end{itemize}
  \item[2.4] Barrieren und Erfolgsfaktoren der Implementierung
  \begin{itemize}
    \item Regulatorische Hürden (Standards, Sicherheit, Genehmigungen)
    \item Technische Anforderungen
    \item Ökonomische Rentabilität
    \item Logistische Koordination
  \end{itemize}
\end{enumerate}

\subsection*{3. Methodik (3--4 Seiten)}
\begin{enumerate}
  \item[3.1] Forschungsdesign: Vergleichende Analyse
  \item[3.2] Benchmarking-Framework (4 Dimensionen)
  \item[3.3] Cost-Benefit-Analysis Modell
  \item[3.4] Datenquellen und Analyserahmen
  \item[3.5] Geografischer und methodischer Scope
\end{enumerate}

\subsection*{4. Ergebnisse (12--15 Seiten)}
\begin{enumerate}
  \item[4.1] Benchmarking-Ergebnisse: Regionale Vergleichsmatrix
  \begin{itemize}
    \item Regulatorische Readiness
    \item Technische Infrastruktur-Status
    \item Kosten-Struktur und Geschäftsmodelle
    \item Implementierungs-Zeithorizonte
  \end{itemize}
  \item[4.2] Cost-Benefit-Analysis
  \begin{itemize}
    \item Szenario-Definition (zentral vs. dezentral, Pipeline vs. Maritime)
    \item NPV-Berechnungen für 2--3 Szenarien pro Region
    \item Break-Even-Analysen
  \end{itemize}
  \item[4.3] Sensitivitätsanalyse
  \begin{itemize}
    \item Einfluss von Subventionen
    \item CO\textsubscript{2}-Preis-Variation
    \item Technologische Kostenreduktionen
  \end{itemize}
\end{enumerate}

\subsection*{5. Diskussion (8--10 Seiten)}
\begin{enumerate}
  \item[5.1] Synthese der Benchmarking-Ergebnisse
  \item[5.2] Interpretation der CBA-Ergebnisse
  \item[5.3] Beantwortung der Forschungsfragen
  \item[5.4] Regionale Implikationen und Trade-offs
  \item[5.5] Limitationen und offene Fragen
\end{enumerate}

\subsection*{6. Fazit (3--4 Seiten)}
\begin{enumerate}
  \item[6.1] Zusammenfassung der Erkenntnisse
  \item[6.2] Handlungsempfehlungen für Entscheidungsträger
  \item[6.3] Ausblick und zukünftige Forschung
\end{enumerate}

\newpage

% ============================================================
% LITERATURVERZEICHNIS
% ============================================================

\section*{Literaturverzeichnis}
\addcontentsline{toc}{section}{Literaturverzeichnis}

\subsection*{Technologie und Infrastruktur}

Hydrogen Council. (2021). \textit{Hydrogen for net-zero growth: A critical price moment}. Hydrogen Council.

International Energy Agency. (2021). \textit{The Future of Hydrogen}. IEA Publications.

Leeuwen, R. P., et al. (2022). Hydrogen infrastructure for the energy transition. \textit{Renewable Energy}, 190, 1--12.

Scapolo, F., \& Smeets, E. (2020). Hydrogen: A promising energy vector. \textit{Energy Policy}, 140, 111319.

\subsection*{Supply-Chain und Logistik}

Christopher, M. (2016). \textit{Logistics \& Supply Chain Management} (5th ed.). Pearson Education.

Pimentel, E., et al. (2019). Hydrogen supply chains: An integrated assessment. \textit{Applied Energy}, 255, 113863.

\subsection*{Regionale Strategien}

Hydrogen Europe. (2022). \textit{Hydrogen Roadmap Europe: A sustainable pathway for the European Energy Transition}.

Ministry of Economy, Trade and Industry Japan. (2021). \textit{Strategic Roadmap for Hydrogen and Fuel Cells}.

Department of Climate Change, Energy, Environment and Water Australia. (2023). Hydrogen Headstart Scheme.

\subsection*{Ökonomische Bewertung}

Schmidt, O., et al. (2019). Green hydrogen from power-to-X: Technology status and perspectives. \textit{Frontiers in Energy Research}, 8, 1--23.

Buttler, A., \& Müller, M. (2021). Making sense of supply roles for hydrogen in Europe's energy transition. \textit{Nature Climate Change}, 11(5), 383--390.

\subsection*{Politische und regulatorische Aspekte}

European Commission. (2020). \textit{A hydrogen strategy for a carbon-neutral Europe}.

Weibel, C., \& Blum, N. U. (2022). Global hydrogen infrastructure development. \textit{Energy Policy}, 165, 112933.

\newpage

% ============================================================
% CONCEPT MAP
% ============================================================

\section*{Concept Map}
\addcontentsline{toc}{section}{Concept Map}

\begin{center}
  \textbf{H\textsubscript{2}-Infrastruktur-Deployment: Synthese der Analyserahmen}

  \vspace{1cm}

  \fbox{
    \begin{minipage}{0.9\textwidth}
      \centering
      \textbf{Energiewende \& Dekarbonisierung} \\
      (Ziel: Schwer-zu-elektrifizieren Sektoren)

      \vspace{0.8cm}
      $\downarrow$

      \vspace{0.8cm}

      \begin{tabular}{c c c}
        \fbox{\textbf{Wasserstoff-}} & \fbox{\textbf{Globale H\textsubscript{2}-}} & \fbox{\textbf{Regionale}} \\
        \fbox{\textbf{Technologie}} & \fbox{\textbf{Infrastruk.-}} & \fbox{\textbf{Strategien}} \\
        & \fbox{\textbf{Lücke}} & \\
        (Produktion, & (Chicken \& & (EU Backbone, \\
        Speicherung, & Egg Problem) & Asien Import, \\
        Transport) & & AU Export) \\
      \end{tabular}

      \vspace{0.8cm}
      $\downarrow$

      \vspace{0.8cm}

      \begin{tabular}{c c c}
        \fbox{\textbf{Barrieren}} & \fbox{\textbf{Erfolgsfaktoren}} & \fbox{\textbf{Implementierungs-}} \\
        (Regulierung, & (Koordination, & (Benchmarking + \\
        Technologie, & Finanzierung, & Cost-Benefit \\
        Ökonomie, & Standards) & Analysis) \\
        Logistik) & & \\
      \end{tabular}

    \end{minipage}
  }
\end{center}

\vspace{1cm}

\noindent \textbf{Methodik-Integration:}
\begin{itemize}
  \item \textbf{Benchmarking:} Vergleicht die 4 Dimensionen systematisch über 3 Regionen
  \item \textbf{CBA (Kostennutzenanalyse):} Quantifiziert wirtschaftliche Rentabilität
  \item \textbf{Sensitivitätsanalyse:} Testet Robustheit gegenüber politischen Parametern
\end{itemize}

\end{document}
